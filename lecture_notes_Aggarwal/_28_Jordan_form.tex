\documentclass{beamer}
\usepackage{amsmath,amsbsy,amsopn,amstext,amsfonts,amssymb}
\usepackage{isomath}
\usepackage{ulem}
%\linespread{1.6}  % double spaces lines
\usepackage{graphicx}
\usepackage{subfigure}
\usepackage{color}
\usepackage{optidef}  % define optimization problems
\usepackage{multicol}  % multiple columns
\usepackage{listings} % for python code
\usepackage{mathrsfs}

\usepackage{polynom}
\newcommand{\adj}{\mathrm{adj}}
\newcommand{\constrainedmin}[3]{
		\begin{mini*}|s|
		{#2}{#1}{}{}
		\addConstraint{#3}
		\end{mini*}
}

\newcommand{\rwbcomment}[1]{{\color{blue}RWB:#1}}
\newcommand{\defeq}{\stackrel{\triangle}{=}}
\newcommand{\abs}[1]{\left|#1\right|}
\newcommand{\norm}[1]{\left\|#1\right\|}
\newcommand{\iprod}[1]{\left<#1\right>}
\newcommand{\ellbf}{\boldsymbol{\ell}}
\newcommand{\nubf}{\boldsymbol{\nu}}
\newcommand{\mubf}{\boldsymbol{\mu}}
\newcommand{\abf}{\mathbf{a}}
\newcommand{\bbf}{\mathbf{b}}
\newcommand{\cbf}{\mathbf{c}}
\newcommand{\dbf}{\mathbf{d}}
\newcommand{\ebf}{\mathbf{e}}
\newcommand{\fbf}{\mathbf{f}}
\newcommand{\gbf}{\mathbf{g}}
\newcommand{\hbf}{\mathbf{h}}
\newcommand{\ibf}{\mathbf{i}}
\newcommand{\jbf}{\mathbf{j}}
\newcommand{\kbf}{\mathbf{k}}
\newcommand{\lbf}{\mathbf{l}}
\newcommand{\mbf}{\mathbf{m}}
\newcommand{\nbf}{\mathbf{n}}
\newcommand{\obf}{\mathbf{o}}
\newcommand{\pbf}{\mathbf{p}}
\newcommand{\qbf}{\mathbf{q}}
\newcommand{\rbf}{\mathbf{r}}
\newcommand{\sbf}{\mathbf{s}}
\newcommand{\tbf}{\mathbf{t}}
\newcommand{\ubf}{\mathbf{u}}
\newcommand{\vbf}{\mathbf{v}}
\newcommand{\wbf}{\mathbf{w}}
\newcommand{\xbf}{\mathbf{x}}
\newcommand{\ybf}{\mathbf{y}}
\newcommand{\zbf}{\mathbf{z}}
\newcommand{\Jbf}{\mathbf{J}}
\newcommand{\Acal}{\mathcal{A}}
\newcommand{\Bcal}{\mathcal{B}}
\newcommand{\Lcal}{\mathcal{L}}
\newcommand{\Ncal}{\mathcal{N}}
\newcommand{\Rcal}{\mathcal{R}}
\definecolor{darkolivegreen}{rgb}{0.33, 0.42, 0.18}

\makeatletter
\newenvironment<>{proofstart}[1][\proofname]{%
    \par
    \def\insertproofname{#1\@addpunct{.}}%
    \usebeamertemplate{proof begin}#2}
  {\usebeamertemplate{proof end}}
\newenvironment<>{proofcont}{%
  \setbeamertemplate{proof begin}{\begin{block}{}}
    \par
    \usebeamertemplate{proof begin}}
  {\usebeamertemplate{proof end}}
\newenvironment<>{proofend}{%
    \par
    \pushQED{\qed}
    \setbeamertemplate{proof begin}{\begin{block}{}}
    \usebeamertemplate{proof begin}}
  {\popQED\usebeamertemplate{proof end}}
\makeatother

\title{ECEn 671: Mathematics of Signals and Systems}
\author{Randal W. Beard}
\institute{Brigham Young University}
\date{\today}

\begin{document}

%-------------------------------
\begin{frame}
	\titlepage
\end{frame}


%%%%%%%%%%%%%%%%%%%%%%%%%%%%%%%%%%%%%%%%%%%%%%%%%%%%%%%%%%%%%%%%%
\section{Jordan Form}
\frame{\sectionpage}	

%----------------------------------
\begin{frame}\frametitle{Jordan Form}
	What if the algebraic multiplicity does not equal the geometric
	multiplicity? (i.e., $q_i \neq m_i$ for some eigenvalue $\lambda_i$ of $A$)?
	
	\vfill
	
	Then we cannot diagonalize $A$ using a similarity transformation.
	However we can ``almost'' diagonalize $A$.
	
	\vfill
	
	The resulting ``almost diagonal`` matrix is called the \underline{Jordan form} of $A$.	
\end{frame}

%----------------------------------
\begin{frame}\frametitle{Jordan Form, cont.}
	
	Suppose the algebraic multiplicity of $\lambda_1$ is $m_1 > 1$ but the
	geometric multiplicity is $q_1 = 1$.
	
	\vfill
	
	Then $\exists$ one linearly independent eigenvector $x_1$ s.t. $Ax_1 = \lambda_1 x_1$.
	
	\vfill
	
	Now form the following chain:
	\begin{align*}
		A\xi_{11} &= \lambda_1 \xi_{11} + x_1\\
		A\xi_{12} &= \lambda_1 \xi_{12} + \xi_{12}\\
		\vdots\\
		A \xi_{1,m_1} &= \lambda_1 \xi_{1,m_1} + \xi_{1,({m_1}-1)}
	\end{align*}
	$\xi_{11} \cdots \xi_{1,m_1}$ are called the ``generalized eigenvectors'' associated with $x_1$.	
\end{frame}

%----------------------------------
\begin{frame}\frametitle{Jordan Form, cont.}
	Note that we can write the generalized eigenvector equations as
	{\footnotesize
	\[ 
		A \begin{pmatrix}
	    	x_1 & \xi_{11} & \cdots & \xi_{1,m_1}
	  	  \end{pmatrix}
		= 
			\begin{pmatrix}
	    		x_1 & \xi_{11} & \cdots & \xi_{1,m_1}
	  		\end{pmatrix}
	  		\underbrace{
				\begin{pmatrix}
	    			\lambda_1 & 1 & \ddots \\
	    			& \lambda_1 & 1 & 0\\
	    			\ddots & & \lambda_1 & \ddots & \ddots \\
	    			& 0 & & \ddots & 1\\
	    			& & \ddots & & \lambda_1\\
	  			\end{pmatrix}
	  		}_{\text{This is called a Jordan block}}
	\]
	}

	\begin{lemma}
		If the geometric multiplicity of $\lambda_i$ is $q_i = 1$ then the associated $m_1-1$ generalized 	eigenvectors are linearly independent of the other eigenvectors.
	\end{lemma}
\end{frame}

%----------------------------------
\begin{frame}\frametitle{Jordan Form, cont.}
	If $1 < q_i < m_i$ then the problem is slightly more complicated. 
	
	\vfill
	
	There are precisely $q_i$ linearly independent eigenvectors associated
	with $\lambda_i$ and there will be $q_i$ Jordan blocks associated with
	$\lambda_i$.  What are the sizes of the Jordan blocks?  For example,
	suppose $m_i = 4$ and $q_i = 2$, the possible Jordan blocks are:
	{\footnotesize
		\[ 
			\begin{pmatrix}
		    	\lambda_1 & 1 & 0\\
		    	0 & \lambda_1 & 1\\
		    	0 & 0 & \lambda_1
		  	\end{pmatrix} 
		  	\text{ and }
			\begin{pmatrix}
		    	\lambda_1
		  	\end{pmatrix}
		  	\text{ i.e., } 
			\begin{pmatrix}
		    	\lambda_1 & 1 & 0 & 0\\
		    	0 & \lambda_1 & 1 & 0\\
		    	0 & 0 & \lambda_1 & 0\\
		    	0 & 0 & 0 & \lambda_1
		  	\end{pmatrix}
		  	\text{ or }
		  	\begin{pmatrix}
		    	\lambda_1 & 0 & 0 & 0\\
		    	0 & \lambda_1 & 1 & 0\\
		    	0 & 0 & \lambda_1 & 1\\
		    	0 & 0 & 0 & \lambda_1
		  	\end{pmatrix}
		\]
	}
	or
	\[ 
		\begin{pmatrix}
	    	\lambda_1 & 1\\
	    	0 & \lambda_1
	  	\end{pmatrix} 
	  	\text{ and }
		\begin{pmatrix}
	    	\lambda_1 & 1\\
	    	0 & \lambda_1
	  	\end{pmatrix}
	  	\text{ i.e., }
	  	\begin{pmatrix}
	    	\lambda_1 & 1 & 0 & 0\\
	    	0 & \lambda_1 & 0 & 0\\
	    	0 & 0 & \lambda_1 & 1\\
	    	0 & 0 & 0 & \lambda_1	  		
	  	\end{pmatrix}
	\]	
	Which option is correct?	
\end{frame}

%----------------------------------
\begin{frame}\frametitle{Jordan Form, cont.}
	To decide, generate the generalized eigenvector for each eigenvector
	and pick the linearly independent ones.
	
	Example:  Let
	\[ 
		A = \begin{pmatrix}
	    		1 & 1 & -1 & 1\\
	    		0 & 1 & 0 & 1\\
	    		0 & 0 & 1 & 1\\
	    		0 & 0 & 0 & 1
	  		\end{pmatrix}
	\]
	Since $\det(\lambda I - A) = (\lambda-1)^4$ we have $\lambda_1 = 1$
	and $m_1 = 4$.
	\[ 
		q_1 = dim(\mathcal{N}
				\begin{pmatrix}
	    			0 & -1 & 1 & -1\\
	    			0 & 0 & 0 & -1\\
	    			0 & 0 & 0 & -1\\
	    			0 & 0 & 0 & 0
	  			\end{pmatrix}
	  		   ) 
	  		= 2
	\] 
	since there are 2 linearly independent rows.
	
\end{frame}

%----------------------------------
\begin{frame}\frametitle{Jordan Form, cont.}
	So there are two linearly independent eigenvectors:
	\[ 
		(\lambda_1I-A)x_1 
			= \begin{pmatrix}
	    		0 & -1 & 1 & -1\\
	    		0 & 0 & 0 & -1\\
	    		0 & 0 & 0 & -1\\
	    		0 & 0 & 0 & 0
	  		  \end{pmatrix}
	  		  \begin{pmatrix}
	    			x_{11}\\x_{12}\\x_{13}\\x_{14}
	  		  \end{pmatrix}
	  		= \begin{pmatrix}
	    		-x_{12} + x_{13} - x_{14}\\-x_{14}\\-x_{14}\\0
	  		  \end{pmatrix}
	  		= \begin{pmatrix}
	    		0\\0\\0\\0
	  		  \end{pmatrix}
	\]
	\[ 
		\implies x_{14} = 0 \text{ and } -x_{12}+x_{13}-x_{14} = 0 
	\]
	so
	\[ 	
		x_1 = \begin{pmatrix}
	    		1\\0\\0\\0
	  		  \end{pmatrix} 
	  	\text{ is an eigenvector, and so is } 
	  	x_2 = \begin{pmatrix}
	    		0\\1\\1\\0
	  		  \end{pmatrix}
	\]	
\end{frame}

%----------------------------------
\begin{frame}\frametitle{Jordan Form, cont.}
	Find the possible generalized eigenvector associated with eigenvector $x_1$:
	\[
		A\xi_{11} = \xi_{11}+x_1 \Rightarrow (\lambda_1 I - A)\xi_{11} = -x_1 
	\]
	\[ 
		\text{i.e. } -\xi_{112} + \xi_{113} - \xi_{114} = 1 \qquad \qquad \xi_{114} = 0 
	\]
	\[ 
		\xi_{112} = \xi_{113} + 1 
		\qquad \text{ so } \qquad 
		\xi_{11} = \begin{pmatrix}1 \\ 2\\ 1 \\ 0 \end{pmatrix} 
		\text{ is valid } 
	\]
	\[ 
		(\lambda_1I-A)\xi_{12} = \xi_{12} 
		\text{ so } 
		\begin{pmatrix}
	    	-\xi_{122} + \xi_{123} - \xi_{124} \\ -\xi_{124} \\ -\xi_{124}
	  	\end{pmatrix}
	  	= \begin{pmatrix}
	    	1\\2\\1\\0
	  	  \end{pmatrix} 
	  	\leftarrow \text{ can't use }. 
	\]	
\end{frame}

%----------------------------------
\begin{frame}\frametitle{Jordan Form, cont.}
	
	{\bf Note:}
		There are an infinite number of possibilities of generalized eigenvectors from each true eigenvector, but you can only pick ones that are linearly independent.  This second eigenvector forms a linearly dependent subset of one of the real eigenvectors.

	\vfill
	
	Therefore, one Jordan block is of size 2.
	
	\vfill
	
	Also solve $(\lambda_1I-A)\xi_{21} = x_2$ i.e.
	\[ 
		\begin{pmatrix}
	    	-\xi_{212} + \xi_{213} - \xi_{214} \\ -\xi_{214} \\ -\xi_{214} \\ 0
	  	\end{pmatrix}
	  	= \begin{pmatrix}
	    	0\\1\\1\\0
	  	  \end{pmatrix}
	  	\Rightarrow \xi_{214} = 1, \xi_{213} = \xi_{212} + 1 
	\]
	so 
	\(
		\xi_{21} = \begin{pmatrix} 0\\1\\2\\1 \end{pmatrix}.
	\)
\end{frame}

%----------------------------------
\begin{frame}\frametitle{Jordan Form, cont.}
	In summary
	\[ 
		A
		\underbrace{
			\begin{pmatrix}
	    		x_1 & \xi_{11} & x_2 & \xi_{21}
	    	\end{pmatrix}
	    }_{S} 
	    = 
	    \underbrace{
	    	\begin{pmatrix}
	    		x_1 & \xi_{11} & x_2 & \xi_{21}
	  		\end{pmatrix}
	  	}_{S}
	  	\underbrace{
	  		\begin{pmatrix}
	  			1& 1 & 0 & 0\\
	  			0 & 1 & 0 & 0\\
	  			0 & 0 & 1 & 1\\
	  			0 & 0 & 0 & 1
			\end{pmatrix}
		}_{J} 
	\]
	or
	\[ 
		A = SJS^{-1} 
	\]
	$J$ is called the ``Jordan'' form of $A$
	
	\vfill
	
	If the eigenvalues are distinct or $q_i = m_i$ for each $i$ then $J =
	\Lambda$ (is diagonal).
	
	\vfill
	
	Otherwise $J$ is block diagonal with Jordan blocks along the diagonal
	($q_i$ Jordan blocks for each eigenvalue).	
\end{frame}

%----------------------------------
\begin{frame}\frametitle{Jordan Form, cont.}
	
	Example:  suppose there are 3 eigenvalues with $\lambda_1 = 1,
	\lambda_2 = 2, \lambda_3 = 3$, and $m_1=1, m_2=2, m_3=3$, and $q_1 =
	1, q_2 = 1, q_3 = 2$.
	There are two possible Jordan forms:
	\[
	\begin{pmatrix}
	    \lambda_1\\
	    & \lambda_2 & 1 && 0\\
	    & & \lambda_2 & &\\
	    &  & & \lambda_3 & 1\\
	    & 0& & & \lambda_3\\
	    & & & & & \lambda_3
	  \end{pmatrix} \text{ or } 
	\begin{pmatrix}
	    \lambda_1\\
	    & \lambda_2 & 1 && 0\\
	    & & \lambda_2 & & \\
	    & & & \lambda_3\\
	    &0 & & & \lambda_3 & 1\\
	    & & & & & \lambda_3
	  \end{pmatrix}
	\]
\end{frame}


\end{document}