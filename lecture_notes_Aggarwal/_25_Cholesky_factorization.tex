\documentclass{beamer}
\usepackage{amsmath,amsbsy,amsopn,amstext,amsfonts,amssymb}
\usepackage{isomath}
\usepackage{ulem}
%\linespread{1.6}  % double spaces lines
\usepackage{graphicx}
\usepackage{subfigure}
\usepackage{color}
\usepackage{optidef}  % define optimization problems
\usepackage{multicol}  % multiple columns
\usepackage{listings} % for python code
\usepackage{mathrsfs}

\usepackage{polynom}
\newcommand{\adj}{\mathrm{adj}}
\newcommand{\constrainedmin}[3]{
		\begin{mini*}|s|
		{#2}{#1}{}{}
		\addConstraint{#3}
		\end{mini*}
}

\newcommand{\rwbcomment}[1]{{\color{blue}RWB:#1}}
\newcommand{\defeq}{\stackrel{\triangle}{=}}
\newcommand{\abs}[1]{\left|#1\right|}
\newcommand{\norm}[1]{\left\|#1\right\|}
\newcommand{\iprod}[1]{\left<#1\right>}
\newcommand{\ellbf}{\boldsymbol{\ell}}
\newcommand{\nubf}{\boldsymbol{\nu}}
\newcommand{\mubf}{\boldsymbol{\mu}}
\newcommand{\abf}{\mathbf{a}}
\newcommand{\bbf}{\mathbf{b}}
\newcommand{\cbf}{\mathbf{c}}
\newcommand{\dbf}{\mathbf{d}}
\newcommand{\ebf}{\mathbf{e}}
\newcommand{\fbf}{\mathbf{f}}
\newcommand{\gbf}{\mathbf{g}}
\newcommand{\hbf}{\mathbf{h}}
\newcommand{\ibf}{\mathbf{i}}
\newcommand{\jbf}{\mathbf{j}}
\newcommand{\kbf}{\mathbf{k}}
\newcommand{\lbf}{\mathbf{l}}
\newcommand{\mbf}{\mathbf{m}}
\newcommand{\nbf}{\mathbf{n}}
\newcommand{\obf}{\mathbf{o}}
\newcommand{\pbf}{\mathbf{p}}
\newcommand{\qbf}{\mathbf{q}}
\newcommand{\rbf}{\mathbf{r}}
\newcommand{\sbf}{\mathbf{s}}
\newcommand{\tbf}{\mathbf{t}}
\newcommand{\ubf}{\mathbf{u}}
\newcommand{\vbf}{\mathbf{v}}
\newcommand{\wbf}{\mathbf{w}}
\newcommand{\xbf}{\mathbf{x}}
\newcommand{\ybf}{\mathbf{y}}
\newcommand{\zbf}{\mathbf{z}}
\newcommand{\Jbf}{\mathbf{J}}
\newcommand{\Acal}{\mathcal{A}}
\newcommand{\Bcal}{\mathcal{B}}
\newcommand{\Lcal}{\mathcal{L}}
\newcommand{\Ncal}{\mathcal{N}}
\newcommand{\Rcal}{\mathcal{R}}
\definecolor{darkolivegreen}{rgb}{0.33, 0.42, 0.18}

\makeatletter
\newenvironment<>{proofstart}[1][\proofname]{%
    \par
    \def\insertproofname{#1\@addpunct{.}}%
    \usebeamertemplate{proof begin}#2}
  {\usebeamertemplate{proof end}}
\newenvironment<>{proofcont}{%
  \setbeamertemplate{proof begin}{\begin{block}{}}
    \par
    \usebeamertemplate{proof begin}}
  {\usebeamertemplate{proof end}}
\newenvironment<>{proofend}{%
    \par
    \pushQED{\qed}
    \setbeamertemplate{proof begin}{\begin{block}{}}
    \usebeamertemplate{proof begin}}
  {\popQED\usebeamertemplate{proof end}}
\makeatother

\title{ECEn 671: Mathematics of Signals and Systems}
\author{Randal W. Beard}
\institute{Brigham Young University}
\date{\today}

\begin{document}

%-------------------------------
\begin{frame}
	\titlepage
\end{frame}

%%%%%%%%%%%%%%%%%%%%%%%%%%%%%%%%%%%%%%%%%%%%%%%%%%%%%%%%%%%%%%%%%
\section{Cholesky Factorization}
\frame{\sectionpage}

%----------------------------------
\begin{frame}\frametitle{Square Root of a Matrix}
	\begin{itemize}
		\item If $B = B^H > 0$ then we can compute the ``square root'' of $B$ as $B = QQ^H$ where $Q = B^{\frac{1}{2}}$ is the square root of $B$.
		\item In general, the square root of a matrix is not unique!
	\end{itemize}
	\begin{example}
		Let $B = 
		\begin{pmatrix}
			9 & 0\\
			0 & 0
		\end{pmatrix}$
	
	
		We can write
		\[ 
			B = \begin{pmatrix} 3 \\ 0 \end{pmatrix}\begin{pmatrix} 3 & 0 \end{pmatrix} 
		\]
		and
		\[ 
			B = \begin{pmatrix}
					3 & 0\\
					0 & 0
				\end{pmatrix}
				\begin{pmatrix}
					3 & 0\\
					0 & 0
				\end{pmatrix}
		\]
		So both $Q = \begin{pmatrix} 3 \\ 0 \end{pmatrix}$ and $Q = \begin{pmatrix}
			3 & 0\\
			0 & 0
		\end{pmatrix}$ are square roots of $B$.
	\end{example}
\end{frame}

%----------------------------------
\begin{frame}\frametitle{Cholesky Factorization}

	\begin{definition}
	The \underline{Cholesky factorization} of $B$ is a square lower triangular square root $L \in \mathbb{C}^{n \times n}$ of $B$, where 
	\[ B = LL^H. \] 
	\end{definition}

	\vfill
	
	Note that this can also be written as
	\[ B=U^HU \]
	where $U=L^H$ is upper triangular.
\end{frame}

%----------------------------------
\begin{frame}\frametitle{Cholesky Factorization: Numerical Algorithm}
	Let $B = \begin{pmatrix}
		\alpha & \vbf^H \\
		\vbf & B_1
	\end{pmatrix}$. 
	Then factor $B$ as
	\begin{align*}
		B &= \begin{pmatrix}
			\alpha & \vbf^H\\
			\vbf & B_1
			\end{pmatrix} \\
		&=
			\begin{pmatrix}
				\sqrt{\alpha} & 0\\
				\frac{\vbf}{\sqrt{\alpha}} & I_{n-1}
			\end{pmatrix}
			\begin{pmatrix}
				1 & 0\\
				0 & B_1-\frac{\vbf\vbf^H}{\alpha}
			\end{pmatrix}
			\begin{pmatrix}
				\sqrt{\alpha} & \frac{\vbf^H}{\sqrt{\alpha}}\\
				0 & I_{n-1}
			\end{pmatrix}\\
%		&=
%			\begin{pmatrix}
%				\sqrt{\alpha} & 0\\
%				\frac{V}{\sqrt{\alpha}} & B_1 - \frac{vv^H}{\alpha}
%			\end{pmatrix}
%			\begin{pmatrix}
%				\sqrt{\alpha} & \frac{V^H}{\sqrt{\alpha}}\\
%				0 & I_{n-1}
%			\end{pmatrix} \\
%		&= 
%			\begin{pmatrix}
%				\alpha & V^H\\
%				V & B_1
%			\end{pmatrix}
	\end{align*}
	
\end{frame}

%----------------------------------
\begin{frame}\frametitle{Cholesky Factorization: Numerical Algorithm, cont.}
	(RECURSIVE ALGORITHM) \\
	Now find the Cholesky factorization of $B_1 - \frac{\vbf\vbf^H}{\alpha} \defeq  G_1G_1^H$, so that
	\begin{align*}
		B &= 
			\begin{pmatrix}
				\sqrt{\alpha} & 0\\
				\frac{\vbf}{\sqrt{\alpha}} & I_{n-1}
			\end{pmatrix}
			\begin{pmatrix}
				1 & 0\\
				0 & G_1G_1^H
			\end{pmatrix}
			\begin{pmatrix}
				\sqrt{\alpha} & \frac{\vbf^H}{\sqrt{\alpha}}\\
				0 & I_{n-1}
			\end{pmatrix} \\
		 &=
			\begin{pmatrix}
				\sqrt{\alpha} & 0\\
				\frac{\vbf}{\sqrt{\alpha}} & G_1
			\end{pmatrix}
			\begin{pmatrix}
				\sqrt{\alpha} & \frac{\vbf^H}{\sqrt{\alpha}}\\
				0 & G_1^H
			\end{pmatrix}
	\end{align*}
	which implies that the Cholesky factor is
	\[ 
		L = \begin{pmatrix}
				\sqrt{\alpha} & 0\\
				\frac{\vbf}{\sqrt{\alpha}} & G_1
			\end{pmatrix}.
	\]
\end{frame}

%----------------------------------
\begin{frame}\frametitle{Cholesky Factorization: Example}
	Let 
	\(
		B = \begin{pmatrix} 
				1 & 2 & 4 & 1 \\
				2 & 13 & 17 & 8 \\
				4 & 17 & 29 & 16 \\
				1 & 8 & 16 & 30
 			\end{pmatrix}.
 	\)
 	Then 
 	\[
 		B = \begin{pmatrix}
 				\alpha_1 & \vbf_1^\top \\ 
 				\vbf_1 & B_1
 			\end{pmatrix},
 	\]
 	where
 	\begin{align*}
 		\alpha_1 &= 1 \\
 		\vbf_1 &= \begin{pmatrix} 2 & 4 & 1 \end{pmatrix}^\top \\
 		B_1 &= \begin{pmatrix}
 					13 & 17 & 8 \\
 					17 & 29 & 16 \\
 					8 & 16 & 30	
 				\end{pmatrix}.
 	\end{align*}
\end{frame}

%----------------------------------
\begin{frame}\frametitle{Cholesky Factorization: Example, cont.}
	Therefore
	\begin{align*}
		B &= \begin{pmatrix}
 				\sqrt{\alpha_1} & 0^\top \\ \frac{\vbf_1}{\sqrt{\alpha_1}} & G_1
 			 \end{pmatrix}
 			 \begin{pmatrix}
 				\sqrt{\alpha_1} & \frac{\vbf_1^\top}{\sqrt{\alpha_1}} \\ 0 & G_1^\top
 			 \end{pmatrix} \\
 		&= \begin{pmatrix}
				1 & 0 & 0 & 0 \\
				2 &   &   &   \\
				4 &   & G_1 & \\
				1 &   &   & 
			\end{pmatrix}
			\begin{pmatrix}
				1 & 2 & 4 & 1 \\
				0 &   &   &   \\
				0 &   & G_1^\top & \\
				0 &   &   & 
			\end{pmatrix}
	\end{align*}
	where
	\begin{align*}
		G_1 G_1^\top &= B_1 - \frac{\vbf_1 \vbf_1^\top}{\alpha_1} \\
		&= \begin{pmatrix}
 				13 & 17 & 8 \\
 				17 & 29 & 16 \\
 				8 & 16 & 30	
 			\end{pmatrix}
 			- \frac{1}{1}\begin{pmatrix} 2 \\ 4 \\ 1 \end{pmatrix}
 			  \begin{pmatrix} 2 & 4 & 1 \end{pmatrix} \\
 		&= 	\begin{pmatrix}
 				9 & 9 & 6 \\ 9 & 13 & 12 \\ 6 & 12 & 29
 			\end{pmatrix}.
	\end{align*}
\end{frame}

%----------------------------------
\begin{frame}\frametitle{Cholesky Factorization: Example, cont.}
	Therefore
	\begin{align*}
		G_1 G_1^\top &= \begin{pmatrix}
 				\sqrt{\alpha_2} & 0^\top \\ \frac{\vbf_2}{\sqrt{\alpha_2}} & G_2
 			 \end{pmatrix}
 			 \begin{pmatrix}
 				\sqrt{\alpha_2} & \frac{\vbf_2^\top}{\sqrt{\alpha_2}} \\ 0 & G_2^\top
 			 \end{pmatrix} \\
 		&= \begin{pmatrix}
				3 & 0 & 0  \\
				3 &  &  \\
				2 &   & G_2 
			\end{pmatrix}
			\begin{pmatrix}
				3 & 3 & 2 \\
				0 &  & \\
				0 &   &  G_2^\top 
			\end{pmatrix}
	\end{align*}
	where
	\begin{align*}
		G_2 G_2^\top &= B_2 - \frac{\vbf_2 \vbf_2^\top}{\alpha_2} \\
		&= \begin{pmatrix}
 				13 & 12 \\
 				12 & 29  
 			\end{pmatrix}
 			- \frac{1}{9} \begin{pmatrix} 9 \\ 6 \end{pmatrix}
 			  \begin{pmatrix} 9 & 6 \end{pmatrix} \\
 		&= 	\begin{pmatrix}
 				4 & 6 \\ 6 & 25
 			\end{pmatrix}.
	\end{align*}
\end{frame}

%----------------------------------
\begin{frame}\frametitle{Cholesky Factorization: Example, cont.}
	Therefore
	\begin{align*}
		G_2 G_2^\top &= \begin{pmatrix}
 				\sqrt{\alpha_3} & 0^\top \\ \frac{\vbf_3}{\sqrt{\alpha_3}} & G_3
 			 \end{pmatrix}
 			 \begin{pmatrix}
 				\sqrt{\alpha_3} & \frac{\vbf_3^\top}{\sqrt{\alpha_3}} \\ 0 & G_3^\top
 			 \end{pmatrix} \\
 		&= \begin{pmatrix}
				2 & 0   \\
				3 & G_3 
			\end{pmatrix}
			\begin{pmatrix}
				2 & 3  \\
				0 & G_3^\top 
			\end{pmatrix}
	\end{align*}
	where
	\begin{align*}
		G_3 G_3^\top &= B_3 - \frac{\vbf_3 \vbf_3^\top}{\alpha_3} \\
		&= 25 - \frac{1}{4} 3 \cdot 3 \\
 		&= 	16
	\end{align*}
	Therefore $G_3=4$.
\end{frame}

%----------------------------------
\begin{frame}\frametitle{Cholesky Factorization: Example, cont.}
	Combining gives
	{\footnotesize
		\begin{align*}
		L 	&= \begin{pmatrix}
					1 & 0 & 0 & 0 \\
					2 &   &   &   \\
					4 &   & G_1 & \\
					1 &   &   & 
				\end{pmatrix} \\
		 	&= \begin{pmatrix}
					1 & 0 & 0 & 0 \\
					2 & 3 & 0 & 0  \\
					4 & 3 &  &  \\
					1 & 2 &   & G_2 
				\end{pmatrix} \\
		 	&= \begin{pmatrix}
					1 & 0 & 0 & 0 \\
					2 & 3 & 0 & 0  \\
					4 & 3 & 2 & 0   \\
					1 & 2 & 3 & G_3 
				\end{pmatrix} \\			 						
			&= \begin{pmatrix}
					1 & 0 & 0 & 0\\
					2 & 3 & 0 & 0\\
					4 & 3 & 2 & 0\\
					1 & 2 & 3 & 4
				\end{pmatrix}.
		\end{align*}
	}
\end{frame}

%----------------------------------
\begin{frame}\frametitle{Applications of Cholesky Factorization: Quadratic Forms}

	The quadratic form
	\[ 
		x^H Q x = \norm{x }_Q^2
	\]
	where $Q=Q^H$, can be written as
	\[ 
		x^H Qx = x^H U^H Ux = \norm{Ux }_2^2 
	\]
	where $Q = U^HU = LL^H$
	
	\vfill
	In other words, work with the regular 2-norm as opposed to the $Q$ norm.
\end{frame}

%----------------------------------
\begin{frame}\frametitle{Applications of Cholesky Factorization: Simulating a random vector}

	Suppose you want to generate in Matlab/Simulink/Python/etc. a Gaussian random vector with covariance $R = R^T>0$.
	
	\vfill

	The Matlab \texttt{randn([m,1])} command returns an $m\times 1$ random vector which is normally distributed with zero mean and co-variance $I$ ($\mathcal{N}(0,I)$).  
	
	\vfill
	
	To generate $\mathcal{N}(0,R)$ let $R = LL^T$ and let $z = Lx$ where $x \sim \mathcal{N}(0,I)$. 
	
	\vfill 

	Then 
	\begin{align*}
		& E\{zz^T\} = E\{Lxx^TL^T\} = LE\{xx^T\}L^T = LL^T = R \\
		\Rightarrow & z \sim \mathcal{N}(0,R).
	\end{align*}

\end{frame}

%----------------------------------
\begin{frame}\frametitle{Applications of Cholesky Factorization: Solving normal equations}

	Normal equations are given by
	\[
		R\mathbf{c} = \mathbf{b}
	\]
	where $R=R^H$ is the Grammian and full rank if the data vectors are linearly independent.
	
	\vfill
	
	Let $R = LL^H$, then $LL^H \mathbf{c} = \mathbf{b}$
	
	\vfill

	First solve 
	\[ 
	L \mathbf{y} = \mathbf{b}
	\]
	by forward substitution, and then solve
	\[ 
	L^H \mathbf{c} = \mathbf{y}
	\]
	by backward substitution.
\end{frame}

%----------------------------------
\begin{frame}\frametitle{Applications of Cholesky Factorization: Kalman filtering}
	In Kalman filtering we propagate two items;  The estimate $\hat{x}(k)$ and the error covariance $P(k)$ where $P(k)=P^T(k) > 0$.
	
	\vfill

	If implemented directly, numerical error can cause $P(k)$ to become indefinite introducing large errors into the estimate $\hat{x}(k)$.
	
	\vfill

	To avoid this problem a ``square root'' Kalman filter is usually implemented where $P(k) = L(k)L^T(k)$ and $L(k)$ is propagated instead of $P(k)$.  Then even with numerical errors in $L(k)$, $P(k)$ is still symmetric positive definite.

\end{frame}

%----------------------------------
\begin{frame}[fragile]\frametitle{Cholesky Factorization: cont.}
In Matlab:
	
\begin{lstlisting}[language=Matlab]
L1 = [2, 0, 0; 3, 4, 0; 5, 6, 7];
A = L1 * L1';
L = chol(A)'	
\end{lstlisting}

\vfill
	
In Python:

\begin{lstlisting}[language=Python]
import numpy as np
import scipy.linalg as linalg
		
L1 = np.array([[2, 0, 0], [3, 4, 0], [5, 6, 7]])
A = L1 @ L1.T
L = linalg.cholesky(A)
\end{lstlisting}	

$L$ should equal $L_1$.

\vfill

Note that both Matlab and Python return an upper triangular matrix.

\vfill

{\bf Homework problem:}  Write your own custom \texttt{cholesky} function and compare to the built in \texttt{cholesky} function on 100 randomly generated symmetric matrices.
\end{frame}




\end{document}