\documentclass{beamer}
\usepackage{amsmath,amsbsy,amsopn,amstext,amsfonts,amssymb}
\usepackage{isomath}
\usepackage{ulem}
%\linespread{1.6}  % double spaces lines
\usepackage{graphicx}
\usepackage{subfigure}
\usepackage{color}
\usepackage{optidef}  % define optimization problems
\usepackage{multicol}  % multiple columns
\usepackage{listings} % for python code
\usepackage{mathrsfs}

\usepackage{polynom}
\newcommand{\adj}{\mathrm{adj}}
\newcommand{\constrainedmin}[3]{
		\begin{mini*}|s|
		{#2}{#1}{}{}
		\addConstraint{#3}
		\end{mini*}
}

\newcommand{\rwbcomment}[1]{{\color{blue}RWB:#1}}
\newcommand{\defeq}{\stackrel{\triangle}{=}}
\newcommand{\abs}[1]{\left|#1\right|}
\newcommand{\norm}[1]{\left\|#1\right\|}
\newcommand{\iprod}[1]{\left<#1\right>}
\newcommand{\ellbf}{\boldsymbol{\ell}}
\newcommand{\nubf}{\boldsymbol{\nu}}
\newcommand{\mubf}{\boldsymbol{\mu}}
\newcommand{\abf}{\mathbf{a}}
\newcommand{\bbf}{\mathbf{b}}
\newcommand{\cbf}{\mathbf{c}}
\newcommand{\dbf}{\mathbf{d}}
\newcommand{\ebf}{\mathbf{e}}
\newcommand{\fbf}{\mathbf{f}}
\newcommand{\gbf}{\mathbf{g}}
\newcommand{\hbf}{\mathbf{h}}
\newcommand{\ibf}{\mathbf{i}}
\newcommand{\jbf}{\mathbf{j}}
\newcommand{\kbf}{\mathbf{k}}
\newcommand{\lbf}{\mathbf{l}}
\newcommand{\mbf}{\mathbf{m}}
\newcommand{\nbf}{\mathbf{n}}
\newcommand{\obf}{\mathbf{o}}
\newcommand{\pbf}{\mathbf{p}}
\newcommand{\qbf}{\mathbf{q}}
\newcommand{\rbf}{\mathbf{r}}
\newcommand{\sbf}{\mathbf{s}}
\newcommand{\tbf}{\mathbf{t}}
\newcommand{\ubf}{\mathbf{u}}
\newcommand{\vbf}{\mathbf{v}}
\newcommand{\wbf}{\mathbf{w}}
\newcommand{\xbf}{\mathbf{x}}
\newcommand{\ybf}{\mathbf{y}}
\newcommand{\zbf}{\mathbf{z}}
\newcommand{\Jbf}{\mathbf{J}}
\newcommand{\Acal}{\mathcal{A}}
\newcommand{\Bcal}{\mathcal{B}}
\newcommand{\Lcal}{\mathcal{L}}
\newcommand{\Ncal}{\mathcal{N}}
\newcommand{\Rcal}{\mathcal{R}}
\definecolor{darkolivegreen}{rgb}{0.33, 0.42, 0.18}

\makeatletter
\newenvironment<>{proofstart}[1][\proofname]{%
    \par
    \def\insertproofname{#1\@addpunct{.}}%
    \usebeamertemplate{proof begin}#2}
  {\usebeamertemplate{proof end}}
\newenvironment<>{proofcont}{%
  \setbeamertemplate{proof begin}{\begin{block}{}}
    \par
    \usebeamertemplate{proof begin}}
  {\usebeamertemplate{proof end}}
\newenvironment<>{proofend}{%
    \par
    \pushQED{\qed}
    \setbeamertemplate{proof begin}{\begin{block}{}}
    \usebeamertemplate{proof begin}}
  {\popQED\usebeamertemplate{proof end}}
\makeatother

\title{ECEn 671: Mathematics of Signals and Systems}
\author{Randal W. Beard}
\institute{Brigham Young University}
\date{\today}

\begin{document}

%-------------------------------
\begin{frame}
	\titlepage
\end{frame}



%%%%%%%%%%%%%%%%%%%%%%%%%%%%%%%%%%%%%%%%%%%%%%%%%%%%%%%%%%%%%%%%%
\section{Neumann Expansion}
\frame{\sectionpage}

%----------------------------------
\begin{frame}\frametitle{Geometric Series}
	One of the most important series in analysis is the geometric series
	\[ 
	S = 1 + x + x^2 + \ldots = \sum_{i=0}^{\infty}x^i 
	\]
	Noting that 
	\begin{align*}
		& 1 + xS = 1 + x + x^2 + \ldots = S \\
		\Rightarrow \qquad & S(1-x) = 1 \\
	\end{align*}
	Therefore 
	\[
	S = \sum_{i=0}^{\infty} x^i = \frac{1}{1-x} = (1-x)^{-1} 
	\]

	\vfill
	
	The series converges if $|x| < 1$.
\end{frame}

%----------------------------------
\begin{frame}\frametitle{Geometric Series for Operators (Neumann Expansion)}
	For operators we have a similar expression:
	\begin{theorem}[Moon Theorem 4.3]
	Suppose $\norm{ \cdot }$ is a norm satisfying the submultiplicative property and $\norm{\Acal} < 1$.  Then $(I-\Acal)^{-1}$ exists and
	\begin{align*}
		(I-\Acal)^{-1} &= \sum_{i=0}^{\infty} \Acal^i 
			= I + \Acal + \Acal^2 + \Acal^3 + \ldots\\
		\text{ where } & \\
		\Acal^2 &= \Acal\Acal \\
		\Acal^3 &= \Acal\Acal^2 \\
		\Acal^k &= \Acal\Acal^{k-1}.
	\end{align*}
\end{theorem}
\end{frame}

%----------------------------------
\begin{frame}\frametitle{Neumann Expansion, Proof}
	Suppose that $(I-\Acal)^{-1}$ does not exist.  Then $\Ncal(I-A)$ is non-trivial.
	
	Therefore, $\exists x \neq 0$ such that 
	\begin{align*}
	(I-\Acal)x = 0 \quad &\iff \quad x = \Acal x \\
	&\iff \quad \norm{ x } = \norm{\Acal x } \leq \norm{ \Acal }\norm{ x } < \norm{ x },
	\end{align*}
	which is a contradiction.  
	
	\vfill
	
	Therefore $(I-\Acal)^{-1}$ exists.
\end{frame}

%----------------------------------
\begin{frame}\frametitle{Neumann Expansion, cont.}
	Note that $\norm{ \Acal^k } \leq \norm{ \Acal }^{k}$ since $\norm{ \cdot }$ satisfies the submultiplication property. 
	
	\vfill 
	
	Since $\norm{\Acal } < 1$
	\[ 
	\lim_{k\to \infty} \norm{ \Acal^k } = 0 \quad \iff \quad \lim_{k \to \infty}\Acal^k = 0 
	\]
	
	\vfill
	
	Note that
	\[ 
	(I - \Acal) ( I + \Acal + \Acal^2 + \cdots + \Acal^{k-1}) = I-\Acal^k 
	\]
	$k \to \infty$ gives
	\[ 
	(I-\Acal)\left( \sum_{i = 0}^{\infty} \Acal^i \right) = I
	\]
	Therefore 
	\[ 
	\sum_{i=0}^{\infty} \Acal^i = (I-\Acal)^{-1}. 
	\]
\end{frame}


\end{document}