\documentclass{beamer}
\usepackage{amsmath,amsbsy,amsopn,amstext,amsfonts,amssymb}
\usepackage{isomath}
\usepackage{ulem}
%\linespread{1.6}  % double spaces lines
\usepackage{graphicx}
\usepackage{subfigure}
\usepackage{color}
\usepackage{optidef}  % define optimization problems
\usepackage{multicol}  % multiple columns
\usepackage{listings} % for python code
\usepackage{mathrsfs}

\usepackage{polynom}
\newcommand{\adj}{\mathrm{adj}}
\newcommand{\constrainedmin}[3]{
		\begin{mini*}|s|
		{#2}{#1}{}{}
		\addConstraint{#3}
		\end{mini*}
}

\newcommand{\rwbcomment}[1]{{\color{blue}RWB:#1}}
\newcommand{\defeq}{\stackrel{\triangle}{=}}
\newcommand{\abs}[1]{\left|#1\right|}
\newcommand{\norm}[1]{\left\|#1\right\|}
\newcommand{\iprod}[1]{\left<#1\right>}
\newcommand{\ellbf}{\boldsymbol{\ell}}
\newcommand{\nubf}{\boldsymbol{\nu}}
\newcommand{\mubf}{\boldsymbol{\mu}}
\newcommand{\abf}{\mathbf{a}}
\newcommand{\bbf}{\mathbf{b}}
\newcommand{\cbf}{\mathbf{c}}
\newcommand{\dbf}{\mathbf{d}}
\newcommand{\ebf}{\mathbf{e}}
\newcommand{\fbf}{\mathbf{f}}
\newcommand{\gbf}{\mathbf{g}}
\newcommand{\hbf}{\mathbf{h}}
\newcommand{\ibf}{\mathbf{i}}
\newcommand{\jbf}{\mathbf{j}}
\newcommand{\kbf}{\mathbf{k}}
\newcommand{\lbf}{\mathbf{l}}
\newcommand{\mbf}{\mathbf{m}}
\newcommand{\nbf}{\mathbf{n}}
\newcommand{\obf}{\mathbf{o}}
\newcommand{\pbf}{\mathbf{p}}
\newcommand{\qbf}{\mathbf{q}}
\newcommand{\rbf}{\mathbf{r}}
\newcommand{\sbf}{\mathbf{s}}
\newcommand{\tbf}{\mathbf{t}}
\newcommand{\ubf}{\mathbf{u}}
\newcommand{\vbf}{\mathbf{v}}
\newcommand{\wbf}{\mathbf{w}}
\newcommand{\xbf}{\mathbf{x}}
\newcommand{\ybf}{\mathbf{y}}
\newcommand{\zbf}{\mathbf{z}}
\newcommand{\Jbf}{\mathbf{J}}
\newcommand{\Acal}{\mathcal{A}}
\newcommand{\Bcal}{\mathcal{B}}
\newcommand{\Lcal}{\mathcal{L}}
\newcommand{\Ncal}{\mathcal{N}}
\newcommand{\Rcal}{\mathcal{R}}
\definecolor{darkolivegreen}{rgb}{0.33, 0.42, 0.18}

\makeatletter
\newenvironment<>{proofstart}[1][\proofname]{%
    \par
    \def\insertproofname{#1\@addpunct{.}}%
    \usebeamertemplate{proof begin}#2}
  {\usebeamertemplate{proof end}}
\newenvironment<>{proofcont}{%
  \setbeamertemplate{proof begin}{\begin{block}{}}
    \par
    \usebeamertemplate{proof begin}}
  {\usebeamertemplate{proof end}}
\newenvironment<>{proofend}{%
    \par
    \pushQED{\qed}
    \setbeamertemplate{proof begin}{\begin{block}{}}
    \usebeamertemplate{proof begin}}
  {\popQED\usebeamertemplate{proof end}}
\makeatother

\title{ECEn 671: Mathematics of Signals and Systems}
\author{Randal W. Beard}
\institute{Brigham Young University}
\date{\today}

\begin{document}

%-------------------------------
\begin{frame}
	\titlepage
\end{frame}

%%%%%%%%%%%%%%%%%%%%%%%%%%%%%%%%%%%%%%%%%%%%%%%%%%%%%%%%%%%%%%%%%
\section{LU Factorization}
\frame{\sectionpage}


%----------------------------------
\begin{frame}\frametitle{LU Factorization}
	\begin{itemize}
		\item 	Suppose that $A \in \mathbb{C}^{n\times n}$ is full rank.  What is a numerically efficient method for computing the solution to $Ax = b$, i.e. $x = A^{-1}b$?
		\item An explicit formula is:
			\[ x = \frac{adj(A)b}{det(A)} \]
			but this requires numerical computation of determinants.
		\item LU factorization is more efficient.
	\end{itemize}
\end{frame}

%----------------------------------
\begin{frame}\frametitle{LU Factorization: Basic Idea}
	\begin{itemize}
		\item Find a permutation matrix $P$, a lower diagonal matrix with ones on the diagonal $L$, and an upper diagonal matrix $U$ such that
			\[ PA = LU.\]
		\item How?  Will illustrate by example:
	\end{itemize}
\end{frame}

%----------------------------------
\begin{frame}\frametitle{LU Factorization: cont.}
	Let 
	\[ 
	A = \begin{pmatrix} 
    		1 & -2 & 3\\
    		-4 & 5 & -6\\
    		7 & -8 & 9
    	\end{pmatrix}
    \]
	The idea is to perform row reductions to get a triangular matrix.  
	\par\noindent{\bf Key Idea:}  Reduce the row with the largest element.
\end{frame}

%----------------------------------
\begin{frame}\frametitle{LU Factorization: cont.}
	First, permute $A$ to get the third row on top:
	\begin{align*}
		P_{13}A &= \underbrace{\begin{pmatrix}
						0 & 0 & 1\\
						0 & 1 & 0\\
						1 & 0 & 0
					\end{pmatrix}}_{P_{13}}
					\begin{pmatrix}
    					1 & -2 & 3\\
    					-4 & 5 & -6\\
    					7 & -8 & 9
    				\end{pmatrix} \\
    			&= \begin{pmatrix}
    					7 & -8 & 9\\
    					-4 & 5 & -6\\
    					1 & -2 & 3
    				\end{pmatrix}
    \end{align*}
	The idea is that you always want to divide by the largest element (in absolute value) in the row to avoid numerical problems.
\end{frame}

%----------------------------------
\begin{frame}\frametitle{LU Factorization: cont.}
	Now zero out the $-4$ and $1$ by multiplying the first row by $+\frac{4}{7}$ and adding to the second row and multiplying the first row by 
	$-\frac{1}{7}$ and adding to the third row:
	\begin{align*}
		E_1 P_{13}A 
		&= 
			\underbrace{\begin{pmatrix}
    			1 & 0 & 0\\
    			\frac{4}{7} & 1 & 0\\
    			\frac{-1}{7} & 0 & 1
  			\end{pmatrix}}_{E_1}
			\begin{pmatrix}
    			7 & -8 & 9\\
    			-4 & 5 & -6\\
    			1 & -2 & 3
  			\end{pmatrix} \\
		&=
			\begin{pmatrix}
    			7 & -8 & 9\\
    			0 & 0.4286 & -5.4286\\
    			0 & -0.8571 & 2.8571
  			\end{pmatrix}
	\end{align*}
\end{frame}

%----------------------------------
\begin{frame}\frametitle{LU Factorization: cont.}
	Now permute (or ``pivot'') to get the largest (in absolute value) number in the second column in the second row:
	\begin{align*}
		P_{23}E_1P_{13}A &= 
			\underbrace{\begin{pmatrix}
    			1 & 0 & 0\\
    			0 & 0 & 1\\
    			0 & 1 & 0
  			\end{pmatrix}}_{P_{23}}
			\begin{pmatrix}
    			7 & -8 & 9\\
    			0 & 0.4286 & -5.4286\\
    			0 & -0.8571 & 2.8571
  			\end{pmatrix} \\
		&=
			\begin{pmatrix}
    			7 & -8 & 9\\
    			0 & -0.8571 & 2.8571\\
    			0 & 0.4286 & -5.4286
  			\end{pmatrix}
	\end{align*}
\end{frame}


%----------------------------------
\begin{frame}\frametitle{LU Factorization: cont.}
	Zero out the $0.4286$ by multiplying the second row by 
	$\frac{0.4286}{0.8571}$ and adding to the third row:
	\begin{align*}
		E_2 P_{23} E_1 P_{13} A 
		&= 
			\underbrace{
				\begin{pmatrix}
    				1 & 0 & 0 \\
    				0 & 1 & 0 \\
    				0 & \frac{0.4286}{0.8571} & 1
  				\end{pmatrix}
  			}_{E_2}
			\begin{pmatrix}
    			7 & -8 & 9\\
    			0 & -0.8571 & 2.8571\\
    			0 & 0.4286 & -5.4286
  			\end{pmatrix} \\
		&=
			\begin{pmatrix}
   				7 & -8 & 9\\
    			0 & -0.8571 & 2.8571\\
    			0 & 0 & -4
  			\end{pmatrix} \\ 
  		&= U 
	\end{align*}
	Therefore
	\begin{align*}
		A &= (E_2P_{23}E_1P_{13})^{-1}U \\
		  &= P_{13}^{-1}E_1^{-1}P_{23}^{-1}E_2^{-1} U 
	\end{align*}
\end{frame}

%----------------------------------
\begin{frame}\frametitle{LU Factorization: cont.}
	Note that if 
 	\(
	E_1 = \begin{pmatrix}
    	1 & 0 & 0\\
    	\frac{4}{7} & 1 & 0\\
    	-\frac{1}{7} & 0 & 1
  		\end{pmatrix},
  	\)
  	then
	\(
	E_1^{-1} = \begin{pmatrix}
    	1 & 0 & 0\\
    	-\frac{4}{7} & 1 & 0\\
    	\frac{1}{7} & 0 & 1
    	\end{pmatrix}
	\) \\
	since 
	\[ 
	\begin{pmatrix}
    	1 & 0 & 0\\
    	\frac{4}{7} & 1 & 0\\
    	-\frac{1}{7} & 0 & 1
    \end{pmatrix}
    \begin{pmatrix}
    	1 & 0 & 0\\
    	-\frac{4}{7} & 1 & 0\\
    	\frac{1}{7} & 0 & 1
    \end{pmatrix}
    =\begin{pmatrix}
    	1 & 0 & 0\\
    	0 & 1 & 0\\
    	0 & 0 & 1
    \end{pmatrix}
	\]
	
	\vfill
	
	So the inverse of any lower diagonal matrix formed by multiplying and adding rows is found by negating the off-diagonal terms.
	
	\vfill

	Therefore $E_1^{-1}$ and $E_2^{-1}$ are easy to compute.
\end{frame}
	
%----------------------------------
\begin{frame}\frametitle{LU Factorization: cont.}
	Also note that for permutation matrices 
	\[
	P_{ij}^{-1} = P_{ji}
	\]
	since 
	\[
	\underbrace{P_{ij}}_{\text{switch $ij$ rows}}
	\underbrace{P_{ij}^{-1}}_{\text{ switch back}} = I.
	\]

	For example
	\[
	\begin{pmatrix}
    	0 & 0 & 1\\
    	0 & 1 & 0\\
    	1 & 0 & 0
	\end{pmatrix}
	\begin{pmatrix}
    	0 & 0 & 1\\
    	0 & 1 & 0\\
    	1 & 0 & 0
	\end{pmatrix}
	=
	\begin{pmatrix}
    	1 & 0 & 0\\
    	0 & 1 & 0\\
    	0 & 0 & 1
	\end{pmatrix}.
	\]
\end{frame}

%----------------------------------
\begin{frame}\frametitle{LU Factorization: cont.}
	So we have $A = VU$
	where 
	\(
	V = P_{13}E_1^{-1}P_{23}E_2^{-1} = 
	\begin{pmatrix}
    	0.1429 & 1 & 0\\
    	-0.5714 & -0.5 & 1\\
    	1 & 0 & 0
  	\end{pmatrix}
	\)
	
	\vfill
		
	Note that $V$ is not lower triangular but
	\begin{align*}
	L &= P_{23}P_{13}V = P_{23}
		\begin{pmatrix}
    		1 & 0 & 0\\
    		-0.5714 & -0.5 & 1\\
    		0.1429 & 1 & 0
  		\end{pmatrix} \\
		&= 
		\begin{pmatrix}
    		1 & 0 & 0\\
    		0.1429 & 1 & 0\\
    		-0.5714 & -0.5 & 1
  		\end{pmatrix}
	\end{align*}
	is, so
	\(
	P_{23}P_{13}A = P_{23}P_{13}VU.
	\)
	Therefore
	\[ 
	PA = LU 
	\]
	where $P = P_{23}P_{13}$.
\end{frame}

%----------------------------------
\begin{frame}\frametitle{LU Factorization: cont.}
	For our example we have
	{\footnotesize
	\[
		\underbrace{
			\begin{pmatrix}
    			0 & 0 & 1\\
    			1 & 0 & 0\\
    			0 & 1 & 0
  			\end{pmatrix}
  		}_{P}
		\underbrace{
			\begin{pmatrix}
    			1 & -2 & 3\\
    			-4 & 5 & -6\\
    			7 & -8 & 9
			\end{pmatrix}
		}_{A}
		= 
		\underbrace{
			\begin{pmatrix}
    			1 & 0 & 0\\
    			0.1429 & 1 & 0\\
    			-0.5714 & -0.5 & 1
			\end{pmatrix}
		}_{L}
		\underbrace{
			\begin{pmatrix}
				7 & -8 & 9\\
    			0 & -0.8571 & 2.8571\\
    			0 & 0 & -4	
			\end{pmatrix}
		}_{U}
	\]
	}
	
	How do we solve the equation $ Ax = b$?

	Suppose 
	\(
		b = \begin{pmatrix} 
				1 \\ 2 \\ 3 
			\end{pmatrix}
	\)

	Note that 
	\[ 
		PAx = Pb 
		    = \begin{pmatrix} 
		    	3 \\ 1 \\ 2 
		      \end{pmatrix}
	\]

	So that
	\[ 
		LUx = Pb. 
	\]

\end{frame}

%----------------------------------
\begin{frame}\frametitle{LU Factorization: cont.}
	Let $y = Ux$ then 
	\begin{align*}
		&  	Ly = Pb \\
		\implies \quad & 
			\begin{pmatrix}
    			1 & 0 & 0\\
    			0.1429 & 1 & 0\\
    			-0.5714 & -0.5 & 1
    		\end{pmatrix}
			\begin{pmatrix} 
				y_1 \\ y_2 \\ y_3 
			\end{pmatrix}
			= 
			\begin{pmatrix} 
				3 \\ 1 \\ 2 
			\end{pmatrix} \\
		\implies \quad & 
			\left\{
				\begin{array}{llr}
					y_1 &= 3\\
					y_2 &= 1 - 0.1429y_1 
						& \quad \text{(easy to solve)}\\
					y_3 &= 2 + 0.5714y_1 + 0.5y_2
				\end{array} 
			\right. \\
		\implies \quad & 
			\left\{
				\begin{array}{llr}
					y_1 &= 3\\
					y_2 &= 0.5741 
						& \quad \text{(easy to solve)} \\
					y_3 &= 4
				\end{array}
			\right.
	\end{align*}
\end{frame}

%----------------------------------
\begin{frame}\frametitle{LU Factorization: cont.}
	Next solve $Ux = y$ for $x$:
	\begin{align*}
		& Ux = y \\
		\implies \quad & 	
		\begin{pmatrix}
    		7 & -8 & 9\\
    		0 & -0.8571 & 2.8571\\
    		0 & 0 & -4
    	\end{pmatrix}
    	\begin{pmatrix}
    		x_1 \\ x_2 \\ x_3	
    	\end{pmatrix}
		= 
		\begin{pmatrix}
			3 \\ 0.5714 \\ 4
		\end{pmatrix} \\
		\implies \quad & 
		\left\{
			\begin{array}{rlr}
				-4x_3 &= 4\\
				-0.8571x_2 &= 0.5714 - 2.8571x_2 
					& \quad \text{(easy to solve)}\\
				7x_1 &= 3 + 8x_2 - x_3		
			\end{array} 
		\right. \\
		\implies \quad & 
		\left\{
			\begin{array}{ll}
				x_1 &= -4\\
				x_2 &= -4\\
				x_3 &= -1		
			\end{array}
		\right.
	\end{align*}	
\end{frame}

%----------------------------------
\begin{frame}[fragile]\frametitle{LU Factorization: cont.}
{\small
In Matlab:
	
\begin{lstlisting}[language=Matlab]
    A = [1, 2, 3; 4, 5, 6; 7, 8, 0];
    [L, U, P] = lu(A)	
\end{lstlisting}

\vfill
	
In Python:

\begin{lstlisting}[language=Python]
    import numpy as np
    import scipy.linalg as linalg
		
    A = np.array([[1, 2, 3], [4, 5, 6], [7, 8, 9]])
    P, L, U = linalg.lu(A)
\end{lstlisting}	
}

\vfill

{\bf Homework problem:}  Write your own custom \texttt{lu} function and compare to the built in \texttt{lu} function on 100 randomly generated matrices.
\end{frame}




\end{document}