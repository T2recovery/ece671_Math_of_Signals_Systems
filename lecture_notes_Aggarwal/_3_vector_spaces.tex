\documentclass{beamer}
\usepackage{amsmath,amsbsy,amsopn,amstext,amsfonts,amssymb}
\usepackage{isomath}
\usepackage{ulem}
%\linespread{1.6}  % double spaces lines
\usepackage{graphicx}
\usepackage{subfigure}
\usepackage{color}
\usepackage{optidef}  % define optimization problems
\usepackage{multicol}  % multiple columns
\usepackage{listings} % for python code
\usepackage{mathrsfs}

\usepackage{polynom}
\newcommand{\adj}{\mathrm{adj}}
\newcommand{\constrainedmin}[3]{
		\begin{mini*}|s|
		{#2}{#1}{}{}
		\addConstraint{#3}
		\end{mini*}
}

\newcommand{\rwbcomment}[1]{{\color{blue}RWB:#1}}
\newcommand{\defeq}{\stackrel{\triangle}{=}}
\newcommand{\abs}[1]{\left|#1\right|}
\newcommand{\norm}[1]{\left\|#1\right\|}
\newcommand{\iprod}[1]{\left<#1\right>}
\newcommand{\ellbf}{\boldsymbol{\ell}}
\newcommand{\nubf}{\boldsymbol{\nu}}
\newcommand{\mubf}{\boldsymbol{\mu}}
\newcommand{\abf}{\mathbf{a}}
\newcommand{\bbf}{\mathbf{b}}
\newcommand{\cbf}{\mathbf{c}}
\newcommand{\dbf}{\mathbf{d}}
\newcommand{\ebf}{\mathbf{e}}
\newcommand{\fbf}{\mathbf{f}}
\newcommand{\gbf}{\mathbf{g}}
\newcommand{\hbf}{\mathbf{h}}
\newcommand{\ibf}{\mathbf{i}}
\newcommand{\jbf}{\mathbf{j}}
\newcommand{\kbf}{\mathbf{k}}
\newcommand{\lbf}{\mathbf{l}}
\newcommand{\mbf}{\mathbf{m}}
\newcommand{\nbf}{\mathbf{n}}
\newcommand{\obf}{\mathbf{o}}
\newcommand{\pbf}{\mathbf{p}}
\newcommand{\qbf}{\mathbf{q}}
\newcommand{\rbf}{\mathbf{r}}
\newcommand{\sbf}{\mathbf{s}}
\newcommand{\tbf}{\mathbf{t}}
\newcommand{\ubf}{\mathbf{u}}
\newcommand{\vbf}{\mathbf{v}}
\newcommand{\wbf}{\mathbf{w}}
\newcommand{\xbf}{\mathbf{x}}
\newcommand{\ybf}{\mathbf{y}}
\newcommand{\zbf}{\mathbf{z}}
\newcommand{\Jbf}{\mathbf{J}}
\newcommand{\Acal}{\mathcal{A}}
\newcommand{\Bcal}{\mathcal{B}}
\newcommand{\Lcal}{\mathcal{L}}
\newcommand{\Ncal}{\mathcal{N}}
\newcommand{\Rcal}{\mathcal{R}}
\definecolor{darkolivegreen}{rgb}{0.33, 0.42, 0.18}

\makeatletter
\newenvironment<>{proofstart}[1][\proofname]{%
    \par
    \def\insertproofname{#1\@addpunct{.}}%
    \usebeamertemplate{proof begin}#2}
  {\usebeamertemplate{proof end}}
\newenvironment<>{proofcont}{%
  \setbeamertemplate{proof begin}{\begin{block}{}}
    \par
    \usebeamertemplate{proof begin}}
  {\usebeamertemplate{proof end}}
\newenvironment<>{proofend}{%
    \par
    \pushQED{\qed}
    \setbeamertemplate{proof begin}{\begin{block}{}}
    \usebeamertemplate{proof begin}}
  {\popQED\usebeamertemplate{proof end}}
\makeatother

\title{ECEn 671: Mathematics of Signals and Systems}
\author{Randal W. Beard}
\institute{Brigham Young University}
\date{\today}

\begin{document}

%-------------------------------
\begin{frame}
	\titlepage
\end{frame}



%%%%%%%%%%%%%%%%%%%%%%%%%%%%%%%%%%%%%%%%%%%%%%%%%%%%%%%%%%%%%%%%%%%%%%%
\section{Vector Spaces}
\frame{\sectionpage}

%----------------------------------
\begin{frame}\frametitle{Vector Spaces}


\subsection*{Definition: Field} A \underline{field} is a set of scalars with well defined addition and multiplication operations.

{\bf Example of fields:}  
\begin{itemize}
\item $\mathbb{R}$ with normal addition and multiplication operations
\item $\mathbb{C}$ with complex addition and complex multiplication
\item The set of quaternions, with addition and quaternion multiplication
\item Binary numbers $\{0, 1\}$ where addition is the ``or'' operator and multiplication is the ``and'' operator.
\end{itemize}
\end{frame}

%----------------------------------
\begin{frame}\frametitle{Vector Spaces}

\begin{definition}[Linear Vector Space] A \underline{linear vector space} is a pair $(\mathbb{X},\mathbb{F})$, where  $\mathbb{X}$ is a set of objects, and $\mathbb{F}$ is a field, this is closed under addition and scalar multiplication. i.e., \\
\begin{itemize}
\item $x\in\mathbb{X}, \alpha \in \mathbb{F} \Rightarrow \alpha x \in \mathbb{X}$
\item $x,y \in \mathbb{X} \Rightarrow x+y\in\mathbb{X}$.
\end{itemize}
\end{definition}
By implication
\begin{itemize}
\item $x\in\mathbb{X}, \alpha,\beta\in\mathbb{F} \Rightarrow (\alpha + \beta)x = \alpha x + \beta x \in \mathbb{X}$
\item $x,y\in\mathbb{X}, \alpha\in\mathbb{F} \Rightarrow \alpha(x + y) = \alpha x + \alpha y \in\mathbb{X}$
\item $x,y\in\mathbb{X}, \alpha,\beta\in\mathbb{F} \Rightarrow \alpha x + \beta y \in \mathbb{X}$.
\end{itemize}
\end{frame}

%----------------------------------
\begin{frame}\frametitle{Vector Spaces: Subspace}

\begin{definition}[Subspace] A \underline{subspace} $V \subset \mathbb{X}$ is a subset of $\mathbb{X}$ that is also a linear vector space, in particular it contains zero.	
\end{definition}


\par\noindent{\bf Important property:} A vector space contains a \underline{zero} element.	

\end{frame}

%----------------------------------
\begin{frame}\frametitle{Vector Spaces: Examples}


The following are vector spaces:
\begin{itemize}
\item $(\mathbb{R}^n, \mathbb{R})$, $(\mathbb{C}^n, \mathbb{C})$, $(\mathbb{R}^{m\times n}, \mathbb{R})$, $(C[a,b], \mathbb{R})$, $(\boldsymbol{\ell}^\infty, \mathbb{R})$, $(L^\infty, \mathbb{R})$.
\end{itemize}

The following are NOT vector spaces:
\begin{itemize}
\item 	The set $\mathbb{X}=\mathbb{R} \times [0,2\pi]$, (a cylinder) is not a vector space for any field $\mathbb{F}$.  This is the state space for an inverted pendulum.
\item The set of rotation matrices is not a vector space for any field $\mathbb{F}$.   This is in the configuration space for robots and satellites.\\
\item The set of unit quaternions is not a vector space for any field.  Quaternions are used extensively in robotics, quantum mechanics, and computer graphics.
\item There are many useful spaces that are NOT linear vector spaces.
\end{itemize}
\end{frame}

%----------------------------------
\begin{frame}\frametitle{Vector Spaces: Linear Independence}


Let $S$ be a vector space and let $T \subset S$.  ($T$ may have uncountable infinite members).  $T$ is linearly independent if for each \underline{finite} nonempty subset of $T$. i.e., $\{p_1,\cdots,p_n\}$ where $p_i\in T$, we have that 
\[
c_1p_1+\cdots+c_np_n=0 \qquad \iff \quad c_1=c_2=\cdots=c_n=0.
\]
Otherwise $T$ is linearly dependent.
\end{frame}

%----------------------------------
\begin{frame}\frametitle{Vector Spaces: Linear Independence}

\begin{example}  Let $S=\mathbb{R}^3$ then the set $T=\{ (1,0,0)^\top, (0, 1, 0)^\top \} \subset \mathbb{R}^3$ is linearly independent since
\[
c_1\begin{pmatrix} 1 \\ 0 \\ 0 \end{pmatrix} + c_2\begin{pmatrix} 0 \\ 1 \\ 0 \end{pmatrix} = \begin{pmatrix} c_1 \\ c_2 \\ 0 \end{pmatrix} = \begin{pmatrix} 0 \\ 0 \\ 0 \end{pmatrix}
\]
if and only if $c_1=c_2=0$.  

However, the set $T=\{ (1,1,0)^\top, (2, 2, 0)^\top \} \subset \mathbb{R}^3$ is linearly dependent since
\[
c_1\begin{pmatrix} 1 \\ 1 \\ 0 \end{pmatrix} + c_2\begin{pmatrix} 2 \\ 2 \\ 0 \end{pmatrix} = \begin{pmatrix} c_1+2c_2 \\ c_1+2c_2 \\ 0 \end{pmatrix} = \begin{pmatrix} 0 \\ 0 \\ 0 \end{pmatrix}
\]
when $c_1=-2$ and $c_2=1$ (as only on example).
\end{example}
\end{frame}

%----------------------------------
\begin{frame}\frametitle{Vector Spaces: Span}

\begin{definition}[Span]
	Let $S$ be a vector space, then $\text{span}(T)$ is the set of all linear combinations of $T\subseteq S$.
\end{definition}
 
\begin{example} 
\[
\text{span}\left\{ \left( \begin{array}{c} 1 \\ 1 \end{array} \right) \right\} = \left\{ \begin{pmatrix} \alpha \\ \alpha \end{pmatrix} | \alpha\in\mathbb{R} \right\}
\]
\end{example}
\begin{example} 
\[
\text{span}\left\{ \left( \begin{array}{c} 1\\0 \end{array}\right),\left(\begin{array}{c}0\\1\end{array}\right)\right\}= \left\{\begin{pmatrix} \alpha \\ \beta \end{pmatrix}:\alpha, \beta\in\mathbb{R}\right\} = \mathbb{R}^2.
\]
\end{example}
\end{frame}

%----------------------------------
\begin{frame}\frametitle{Vector Spaces: Basis}

\begin{definition}[Basis] $T$ is a \underline{basis} for the vector space $S$ if $T$ is linearly independent and $\mathrm{span}(T)=S$.	
\end{definition}

\begin{definition}[Dimension] The dimension of the vector space $S$ is the smallest number of linearly independent vectors needed to span $S$.	
\end{definition}

\begin{example}
One possible basis for $\mathbb{R}^n$ is given by
\[
\left\{\left(\begin{array}{c}1\\0\\\vdots\\0\end{array}\right),\left(\begin{array}{c}0\\1\\\vdots\\0\end{array}\right)\cdots,\left(\begin{array}{c}0\\0\\\vdots\\1\end{array}\right)\right\}.
\]
Therefore $\dim(\mathbb{R}^n) = n$.
\end{example}
\end{frame}

%----------------------------------
\begin{frame}\frametitle{Vector Spaces: Basis}

\begin{example}
One possible basis for $\boldsymbol{\ell}^\infty$ is given by
\[
\left\{
  \left(
    \begin{array}{c}
      1\\
      0\\
      0\\
      \vdots\\
      \vdots\\
      \vdots
    \end{array}
  \right)
  ,\left(
    \begin{array}{c}
      0\\
      1\\
      0\\
      \vdots\\
      \vdots\\
      \vdots
    \end{array}
  \right)
  ,\cdots,
  \left(
    \begin{array}{c}
      0\\
      0\\
      \vdots\\
      0\\
      1\\
      0\\
      \vdots
    \end{array}
  \right)
  ,\cdots
\right\}
\]
Therefore $\dim(\boldsymbol{\ell}^\infty) = \infty$.
\end{example}
\end{frame}

%----------------------------------
\begin{frame}\frametitle{Vector Spaces: Basis}
\begin{example}
The set of all polynomials $P$ is a vector space with basis
\[ \{1,t,t^2,\cdots\} \]
Therefore $\dim(P) = \infty$.
\end{example}

\begin{example}
The set of all polynomials of degree $\leq q$ $P^q$ is a vector space with basis
\[ \{1,t,t^2,\cdots, t^q\} \]
Therefore $\dim(P^q) = q$.
\end{example}
\end{frame}


\end{document}