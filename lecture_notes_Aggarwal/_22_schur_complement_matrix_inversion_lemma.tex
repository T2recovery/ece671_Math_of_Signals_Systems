\documentclass{beamer}
\usepackage{amsmath,amsbsy,amsopn,amstext,amsfonts,amssymb}
\usepackage{isomath}
\usepackage{ulem}
%\linespread{1.6}  % double spaces lines
\usepackage{graphicx}
\usepackage{subfigure}
\usepackage{color}
\usepackage{optidef}  % define optimization problems
\usepackage{multicol}  % multiple columns
\usepackage{listings} % for python code
\usepackage{mathrsfs}

\usepackage{polynom}
\newcommand{\adj}{\mathrm{adj}}
\newcommand{\constrainedmin}[3]{
		\begin{mini*}|s|
		{#2}{#1}{}{}
		\addConstraint{#3}
		\end{mini*}
}

\newcommand{\rwbcomment}[1]{{\color{blue}RWB:#1}}
\newcommand{\defeq}{\stackrel{\triangle}{=}}
\newcommand{\abs}[1]{\left|#1\right|}
\newcommand{\norm}[1]{\left\|#1\right\|}
\newcommand{\iprod}[1]{\left<#1\right>}
\newcommand{\ellbf}{\boldsymbol{\ell}}
\newcommand{\nubf}{\boldsymbol{\nu}}
\newcommand{\mubf}{\boldsymbol{\mu}}
\newcommand{\abf}{\mathbf{a}}
\newcommand{\bbf}{\mathbf{b}}
\newcommand{\cbf}{\mathbf{c}}
\newcommand{\dbf}{\mathbf{d}}
\newcommand{\ebf}{\mathbf{e}}
\newcommand{\fbf}{\mathbf{f}}
\newcommand{\gbf}{\mathbf{g}}
\newcommand{\hbf}{\mathbf{h}}
\newcommand{\ibf}{\mathbf{i}}
\newcommand{\jbf}{\mathbf{j}}
\newcommand{\kbf}{\mathbf{k}}
\newcommand{\lbf}{\mathbf{l}}
\newcommand{\mbf}{\mathbf{m}}
\newcommand{\nbf}{\mathbf{n}}
\newcommand{\obf}{\mathbf{o}}
\newcommand{\pbf}{\mathbf{p}}
\newcommand{\qbf}{\mathbf{q}}
\newcommand{\rbf}{\mathbf{r}}
\newcommand{\sbf}{\mathbf{s}}
\newcommand{\tbf}{\mathbf{t}}
\newcommand{\ubf}{\mathbf{u}}
\newcommand{\vbf}{\mathbf{v}}
\newcommand{\wbf}{\mathbf{w}}
\newcommand{\xbf}{\mathbf{x}}
\newcommand{\ybf}{\mathbf{y}}
\newcommand{\zbf}{\mathbf{z}}
\newcommand{\Jbf}{\mathbf{J}}
\newcommand{\Acal}{\mathcal{A}}
\newcommand{\Bcal}{\mathcal{B}}
\newcommand{\Lcal}{\mathcal{L}}
\newcommand{\Ncal}{\mathcal{N}}
\newcommand{\Rcal}{\mathcal{R}}
\definecolor{darkolivegreen}{rgb}{0.33, 0.42, 0.18}

\makeatletter
\newenvironment<>{proofstart}[1][\proofname]{%
    \par
    \def\insertproofname{#1\@addpunct{.}}%
    \usebeamertemplate{proof begin}#2}
  {\usebeamertemplate{proof end}}
\newenvironment<>{proofcont}{%
  \setbeamertemplate{proof begin}{\begin{block}{}}
    \par
    \usebeamertemplate{proof begin}}
  {\usebeamertemplate{proof end}}
\newenvironment<>{proofend}{%
    \par
    \pushQED{\qed}
    \setbeamertemplate{proof begin}{\begin{block}{}}
    \usebeamertemplate{proof begin}}
  {\popQED\usebeamertemplate{proof end}}
\makeatother

\title{ECEn 671: Mathematics of Signals and Systems}
\author{Randal W. Beard}
\institute{Brigham Young University}
\date{\today}

\begin{document}

%-------------------------------
\begin{frame}
	\titlepage
\end{frame}

%%%%%%%%%%%%%%%%%%%%%%%%%%%%%%%%%%%%%%%%%%%%%%%%%%%%%%%%%%%%%%%%%
\section{Schur Complement and the Matrix Inversion Lemma}
\frame{\sectionpage}

%----------------------------------
\begin{frame}\frametitle{Schur Complement}
	\begin{definition}
		Consider the partitioned matrix
		\[
		A = \begin{bmatrix} A_{11} & A_{12} \\ A_{21} & A_{22} \end{bmatrix}.
		\]	
		\begin{enumerate}
			\item When $A_{11}$ is non-singular, 
				\[ S_{ch}(A_{11}) \defeq A_{22} - A_{21}A_{11}^{-1}A_{12} \]
				is called the \underline{Schur Complement of $A_{11}$ in $A$}.
			\item When $A_{22}$ is non-singular, 
				\[ S_{ch}(A_{22}) \defeq A_{11} - A_{12}A_{22}^{-1}A_{21} \]
				is called the \underline{Schur Complement of $A_{22}$ in $A$}.
		\end{enumerate}
	\end{definition}
\end{frame}

%----------------------------------
\begin{frame}\frametitle{Schur Complement, cont.}
	\begin{lemma}
	When $A_{11}$ is nonsingular, $A$ is nonsingular if and only if $S_ch(A_{11})$ is nonsingular, in which case
	\[
	A^{-1} = \begin{bmatrix}
				A_{11}^{-1} + A_{11}^{-1}A_{12}S_{ch}^{-1}(A_{11})A_{21}A_{11}^{-1} &
				-A_{11}^{-1}A_{12}S_{ch}^{-1}(A_{11}) \\
				-S_{ch}^{-1}(A_{11})A_{21}A_{11}^{-1} &
				S_{ch}^{-1}(A_{11})
 			 \end{bmatrix}
	\]	
	\end{lemma}
	\begin{lemma}
	When $A_{22}$ is nonsingular, $A$ is nonsingular if and only if $S_ch(A_{22})$ is nonsingular, in which case
	\[
	A^{-1} = \begin{bmatrix}
				S_{ch}^{-1}(A_{22}) &
				-S_{ch}^{-1}(A_{22})A_{12}A_{22}^{-1} \\
				-A_{22}^{-1}A_{12}S_{ch}^{-1}(A_{22}) &
				A_{22}^{-1} + A_{22}^{-1}A_{21}S_{ch}^{-1}(A_{22})A_{12}A_{22}^{-1} 
 			 \end{bmatrix}
	\]	
	\end{lemma}
	\begin{proof}
	By direct manipulation.	
	\end{proof}
\end{frame}

%----------------------------------
\begin{frame}\frametitle{Matrix Inversion Lemma}
	\begin{lemma}[Matrix Inversion Lemma]
		If $A\in\mathbb{R}^{n\times n}$ and $R\in\mathbb{R}^{m\times m}$ are invertible, and $X\in\mathbb{R}^{n\times m}$ and $Y\in\mathbb{R}^{m\times n}$ then
		\[ (A + X R Y)^{-1} = A^{-1} - A^{-1}X(R^{-1}+YA^{-1}X)^{-1}YA^{-1} \]
	\end{lemma}
	\begin{proof}  Equate the $(2,2)$ elements of $A^{-1}$ in the previous slide, and re-label matrices.	
	\end{proof}
	
\end{frame}

%----------------------------------
\begin{frame}\frametitle{Matrix Inversion Lemma, cont.}
	\begin{itemize}
	\item 	A special case of this matrix inversion lemma is the formula
		\[ 
		(A + xy^H)^{-1} = A^{-1} - \frac{A^{-1}xy^H A^{-1}}{1 + y^H A^{-1}x} 
		\]
		where $x$ and $y$ are vectors.
	\item Sylvester's inequality gives
		\[
		rank(x)+rank(y) - 1 \leq rank(xy^H) \leq min(rank(x),rank(y)).
		\]
		But 
		\begin{align*}
			rank(x)+rank(y) - 1	&= 1 \\
			min(rank(x),rank(y)) &= 1
		\end{align*}
	\item Therefore $rank(xy^H) = 1$
	\end{itemize}
\end{frame}




\end{document}