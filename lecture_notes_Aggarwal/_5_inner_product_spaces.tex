\documentclass{beamer}
\usepackage{amsmath,amsbsy,amsopn,amstext,amsfonts,amssymb}
\usepackage{isomath}
\usepackage{ulem}
%\linespread{1.6}  % double spaces lines
\usepackage{graphicx}
\usepackage{subfigure}
\usepackage{color}
\usepackage{optidef}  % define optimization problems
\usepackage{multicol}  % multiple columns
\usepackage{listings} % for python code
\usepackage{mathrsfs}

\usepackage{polynom}
\newcommand{\adj}{\mathrm{adj}}
\newcommand{\constrainedmin}[3]{
		\begin{mini*}|s|
		{#2}{#1}{}{}
		\addConstraint{#3}
		\end{mini*}
}

\newcommand{\rwbcomment}[1]{{\color{blue}RWB:#1}}
\newcommand{\defeq}{\stackrel{\triangle}{=}}
\newcommand{\abs}[1]{\left|#1\right|}
\newcommand{\norm}[1]{\left\|#1\right\|}
\newcommand{\iprod}[1]{\left<#1\right>}
\newcommand{\ellbf}{\boldsymbol{\ell}}
\newcommand{\nubf}{\boldsymbol{\nu}}
\newcommand{\mubf}{\boldsymbol{\mu}}
\newcommand{\abf}{\mathbf{a}}
\newcommand{\bbf}{\mathbf{b}}
\newcommand{\cbf}{\mathbf{c}}
\newcommand{\dbf}{\mathbf{d}}
\newcommand{\ebf}{\mathbf{e}}
\newcommand{\fbf}{\mathbf{f}}
\newcommand{\gbf}{\mathbf{g}}
\newcommand{\hbf}{\mathbf{h}}
\newcommand{\ibf}{\mathbf{i}}
\newcommand{\jbf}{\mathbf{j}}
\newcommand{\kbf}{\mathbf{k}}
\newcommand{\lbf}{\mathbf{l}}
\newcommand{\mbf}{\mathbf{m}}
\newcommand{\nbf}{\mathbf{n}}
\newcommand{\obf}{\mathbf{o}}
\newcommand{\pbf}{\mathbf{p}}
\newcommand{\qbf}{\mathbf{q}}
\newcommand{\rbf}{\mathbf{r}}
\newcommand{\sbf}{\mathbf{s}}
\newcommand{\tbf}{\mathbf{t}}
\newcommand{\ubf}{\mathbf{u}}
\newcommand{\vbf}{\mathbf{v}}
\newcommand{\wbf}{\mathbf{w}}
\newcommand{\xbf}{\mathbf{x}}
\newcommand{\ybf}{\mathbf{y}}
\newcommand{\zbf}{\mathbf{z}}
\newcommand{\Jbf}{\mathbf{J}}
\newcommand{\Acal}{\mathcal{A}}
\newcommand{\Bcal}{\mathcal{B}}
\newcommand{\Lcal}{\mathcal{L}}
\newcommand{\Ncal}{\mathcal{N}}
\newcommand{\Rcal}{\mathcal{R}}
\definecolor{darkolivegreen}{rgb}{0.33, 0.42, 0.18}

\makeatletter
\newenvironment<>{proofstart}[1][\proofname]{%
    \par
    \def\insertproofname{#1\@addpunct{.}}%
    \usebeamertemplate{proof begin}#2}
  {\usebeamertemplate{proof end}}
\newenvironment<>{proofcont}{%
  \setbeamertemplate{proof begin}{\begin{block}{}}
    \par
    \usebeamertemplate{proof begin}}
  {\usebeamertemplate{proof end}}
\newenvironment<>{proofend}{%
    \par
    \pushQED{\qed}
    \setbeamertemplate{proof begin}{\begin{block}{}}
    \usebeamertemplate{proof begin}}
  {\popQED\usebeamertemplate{proof end}}
\makeatother

\title{ECEn 671: Mathematics of Signals and Systems}
\author{Randal W. Beard}
\institute{Brigham Young University}
\date{\today}

\begin{document}

%-------------------------------
\begin{frame}
	\titlepage
\end{frame}



%%%%%%%%%%%%%%%%%%%%%%%%%%%%%%%%%%%%%%%%%%%%%%%%%%%%%%%%%%%%%%%%%%%%%%%
\section{Inner Product Spaces}
\frame{\sectionpage}

%----------------------------------
\begin{frame}\frametitle{Inner Product Spaces}
\begin{definition}[Inner Product] Let $S$ be a vector space over $\mathbb{R}$.  An inner product $<\cdot,\cdot>:S \times S \to \mathbb{R}$ has the following properties:\\
\begin{flalign*}
(IP1) \qquad &\iprod{x,y} = \overline{\iprod{ y,x }}\\
(IP2) \qquad &\iprod{\alpha x,y} = \alpha \iprod{ x,y }\\
(IP3) \qquad &\iprod{ x+y,z } = \iprod{ x,z } + \iprod{ y, z }\\
(IP4) \qquad &\iprod{ x,x } > 0 \quad \text{~if~} x \neq 0, \iprod{ x,x}=0 \Leftrightarrow x=0
\end{flalign*}
\end{definition}

\begin{definition}[Inner Product Space] A vector space with an inner product defined is called an inner-product space.	
\end{definition}

\begin{definition}[Hilbert Space] A complete inner-product space is called a \underline{Hilbert space}.	
\end{definition}

\end{frame}

%----------------------------------
\begin{frame}\frametitle{Inner Product Spaces: Examples}
\begin{itemize}
	\item $\mathbb{R}^n$: $\iprod{ x,y } = \sum_{i=1}^{n} x_i y_i = x^Ty$ is called the Euclidean inner product.
	\item $\mathbb{C}^n$: $\iprod{ x,y } = \sum_{i=1}^{n} x_i \overline{y_i} = y^Hx$
\end{itemize}
\begin{itemize}
	\item $\mathbb{R}^n$ with the Euclidean inner product is a Hilbert space	.
	\item $\mathbb{C}^n$ with the Euclidean inner product is a Hilbert space.
	\item All finite-dimensional inner-product spaces are Hilbert spaces.
\end{itemize}
\end{frame}
	
%----------------------------------
\begin{frame}\frametitle{Inner Product Spaces: Examples}
\begin{itemize}
	\item Real sequences $\boldsymbol{\ell}_2 $: $\iprod{ x,y }_{\ell_2} = \sum_{i=1}^{\infty} x_i y_i$
	\item Complex sequences $\boldsymbol{\ell}_2$:  $\iprod{ x,y }_{\ell_2} = \sum_{i=1}^{\infty} x_i \overline{y_i}$
	\item Both of these examples are Hilbert spaces.
\end{itemize}
\end{frame}

%----------------------------------
\begin{frame}\frametitle{Inner Product Spaces: Examples}
\begin{itemize}
	\item Complex function space $L_2^n(\Omega)$ with inner product:
		\[
		\iprod{ x,y } = \displaystyle \int_{\infty} y^H(t)x(t) \,dt
		\]
		is a Hilbert space, but 
	\item Continuous function $C[a,b]$ with the same inner product is NOT a Hilbert space.
\end{itemize}
\end{frame}

%----------------------------------
\begin{frame}\frametitle{Norms vs Inner Products}
Every inner product defines a norm (but not vice-versa)
\[ 
\norm{ x}  = \iprod{ x,x }^{\frac{1}{2}} 
\] 
where $\norm{\cdot}$ is called the norm induced by the inner product $\iprod{\cdot, \cdot}$.
\end{frame}

%----------------------------------
\begin{frame}\frametitle{Examples of inducted norms}

\begin{flalign*}
\norm{ \cdot}_2\text{:  }\iprod{ x,x }^{1/2} &= \left( \displaystyle \sum_{i=1}^{n}x_i^2 \right)^{1/2} = \norm{ x}_2\\
\norm{ \cdot}_{\ell_2}\text{:  }\iprod{ x,x }^{1/2} &= \left( \displaystyle \sum_{i=1}^{\infty} x_i^2 \right)^{1/2} = \norm{ x}_{\ell_2}\\
\norm{ \cdot}_{L_2}\text{:  }\iprod{ x,x }^{1/2} &= \left( \displaystyle \int_{\Omega} x^T(t)x(t) dt \right)^{1/2}
= \left( \int_{\Omega} \norm{ x(t)}_2^2 dt \right)^{1/2}
= \norm{ x}_{L_2}\\
\end{flalign*}

Note that induced norms are all $2-$norms.
\end{frame}

%----------------------------------
\begin{frame}\frametitle{Cauchy-Schwartz Inequality}
\begin{theorem}[Cauchy-Schwartz]
Let $S$ be any inner product space (doesn't need to be Hilbert) and let $\norm{ \cdot}  = \iprod{ \cdot, \cdot }^{1/2}$\\
then $\forall x,y \in S$
\[ |\iprod{ x,y } | \leq \norm{ x}  \norm{ y}  \]
with equality iff $y=\alpha x$ where $\alpha\in\mathbb{F}$ is any scalar in the field $\mathbb{F}$.
\end{theorem}
\end{frame}

%----------------------------------
\begin{frame}\frametitle{Cauchy-Schwartz Inequality: Proof}
The inequality clearly holds if either $x=0$ or $y=0$.  Therefore assume that $x \neq 0$ and $y \neq 0$. Then
\begin{align*}
	\norm{ x-\alpha y} ^2 &= \iprod{ x-\alpha y, x - \alpha y } \\
		&=\iprod{ x ,x } - \alpha \iprod{ y,x } - \iprod{ x, \alpha y } + \iprod{ \alpha y, \alpha y } \\
		&=\iprod{ x ,x } - \alpha \iprod{ y,x } - \overline{\overline{\iprod{ x, \alpha y }}} + \alpha \overline{\overline{\iprod{ y, \alpha y }}} \\
		&=\iprod{ x ,x } - \alpha \iprod{ y,x } - \overline{\iprod{ \alpha y, x }} + \alpha \overline{\iprod{ \alpha y, y }} \\
		&=\iprod{ x ,x } - \alpha \iprod{ y,x } - \overline{\alpha} \overline{\iprod{ y, x }} + \alpha\overline{\alpha}\overline{\iprod{ y, \alpha y }}\\
		&=\iprod{ x ,x } - \alpha \iprod{ y,x } - \overline{\alpha} \iprod{ x, y } + \abs{\alpha}^2\iprod{ y, y } \\
		&=\norm{x}^2 - \alpha \iprod{ y,x } - \overline{\alpha} \iprod{ x, y } + \abs{\alpha}^2\norm{y}^2 \\
\end{align*}
\end{frame}

%----------------------------------
\begin{frame}\frametitle{Cauchy-Schwartz Inequality: Proof}
Recall the technique of completing the square:
\begin{flalign*}
ax^2+bx+c &= a(x^2 + \frac{b}{a}x) + c\\
&= a(x+\frac{b}{2a})^2 - \frac{b}{4a}^2 + c.
\end{flalign*}
Complete the square in $\alpha$:
\begin{align*}
\norm{ x-\alpha y}^2 &= \norm{ y} ^2 \left(\alpha\overline{\alpha} - \alpha \frac{\overline{ \iprod{ x,y } }}{\norm{ y} ^2} - \bar\alpha \frac{\iprod{ x,y }}{\norm{ y} ^2} \right) + \norm{ x} ^2 \\
	&= \norm{ y} ^2 \left( \alpha - \frac{\iprod{ x,y }}{\norm{ y} ^2}\right)\left(\bar\alpha - \frac{\overline{\iprod{ x,y }}}{\norm{ y} ^2}\right) - \frac{|\iprod{ x,y } |^2}{\norm{ y} ^2} + \norm{ x} ^2
\end{align*}
\end{frame}

%----------------------------------
\begin{frame}\frametitle{Cauchy-Schwartz Inequality: Proof}
Let $\alpha^*= \frac{\iprod{ x,y }}{\norm{ y} ^2}$ to get
\begin{align*}
	& 0 \leq \norm{ x - \alpha^*y} ^2 = \norm{ x} ^2 - \frac{|\iprod{ x,y }|^2}{\norm{ y} ^2} \\
	\Rightarrow &\qquad |\iprod{ x,y } |^2 \leq \norm{ x} ^2\norm{ y} ^2 \\
	\Rightarrow &\qquad |\iprod{ x,y } | \leq \norm{ x}  \norm{ y}
\end{align*}
\end{frame}


\end{document}